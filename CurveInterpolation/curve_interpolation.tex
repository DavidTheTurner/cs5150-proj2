\documentclass{article}
\usepackage{graphicx} % Required for inserting images
\usepackage{amsmath}
\usepackage{amsthm}
\usepackage{amssymb}
\usepackage{xcolor}


\setlength{\topmargin}{-.5in}
\setlength{\oddsidemargin}{0 in}
\setlength{\evensidemargin}{0 in}
\setlength{\textwidth}{6.5truein}
\setlength{\textheight}{8.5truein}

\renewcommand\qedsymbol{$\blacksquare$}

\begin{document}
    \begin{proof}
        To prove the second derivative at $b_0$ is $6(b_0 - 2b_1 + b_2)$ and
        $b_3$ is $6(b_1 - 2b_2 + b_3)$, we must first find the derivatives of
        \[
            C(t) = (1 - t)^3b_0 + 3(1 - t)^2tb_1 + 3(1 - t)t^2b_2 + t^3b_3
        \]
        For the first derivative, we have
        \begin{align*}
            C'(t) & = 3(1 - t)^2(-1)b_0 + 3(2(1 - t)(-1)t + (1 - t)^2)b_1 + 3((-1)t^2 + 2(1 - t)t)b_2 + 3t^2b_3
            \\ & =
            -3(1 - t)^2b_0 + 3(-2(1 - t)t + (1 - t)^2)b_1 + 3(-t^2 + 2(1 - t)t)b_2 + 3t^2b_3
            \\ & =
            -3(1 - t)^2b_0 - 6(1 - t)tb_1 + 3(1 - t)^2b_1 - 3t^2b_2 + 6(1 - t)tb_2 + 3t^2b_3
        \end{align*}
        Then for the second derivative, we have
        \begin{align*}
            C''(t) & = -3(2(1 - t)(-1))b_0 - 6((1-t) + (-1)t)b_1 + 3(2(1 - t)(-1))b_1 - 6tb_2 +6((-1)t + (1 - t))b_2 + 6tb_3
            \\ & =
            6(1 - t)b_0 - 6(1 - t)b_1 + 6tb_1 - 6(1 - t)b_1 - 6tb_2 - 6tb_2 + 6(1 - t)b_2 + 6tb_3
            \\ & =
            6b_0 - 6tb_0 - 6b_1 + 6tb_1 + 6tb_1 - 6b_1 + 6tb_1 - 12tb_2 + 6b_2 - 6tb_2 + 6tb_3
            \\ & =
            6b_0 - 6tb_0 - 12b_1 + 18tb_1 - 18tb_2 + 6b_2 + 6tb_3
        \end{align*}
        Now, since $C(0) = b_0$ the second derivative at $b_0$ is
        \begin{align*}
            C''(0) & = 6b_0 - 6(0)b_0 - 12b_1 + 18(0)b_1 - 18(0)b_2 + 6b_2 + 6(0)b_3
            \\ & =
            6b_0 - 12b_1 + 6b_2
            \\ & =
            6(b_0 - 2b_1 + b_2)
        \end{align*}
        as desired, and since $C(1) = b_3$, the second derivative at $b_3$ is
        \begin{align*}
            C''(1) & = 6b_0 - 6(1)b_0 - 12b_1 + 18(1)b_1 - 18(1)b_2 + 6b_2 + 6(1)b_3
            \\ & =
            6b_0 - 6b_0 - 12b_1 + 18b_1 - 18b_2 + 6b_2 + 6b_3
            \\ & =
            0 + 6b_1 - 12b_2 + 6b_3
            \\ & =
            6(b_1 - 2b_2 + b_3)
        \end{align*}
        as desired.

        Now, to prove our system becomes
        \[
            \begin{pmatrix}
                4 & 1 & 0 & \cdots & \cdots & 0 \\
                1 & 4 & 1 & 0 & \cdots & 0 \\
                0 & \ddots & \ddots & \ddots & \ddots & \\
                0 & \cdots & \cdots & 1 & 4 & 1 \\
                0 & \cdots & \cdots & \cdots & 1 & 4
            \end{pmatrix}
            \begin{pmatrix}
                d_1 \\
                d_2 \\
                \vdots \\
                d_{N - 2} \\
                d_{N - 1}
            \end{pmatrix}
            =
            \begin{pmatrix}
                6x_1 - x_0 \\
                6x_2 \\
                \vdots \\
                6x_{N - 2} \\
                6x_{N - 1} - x_N \\
            \end{pmatrix}
        \]
        for $N \geq 4$. Given equation $(1)$, we know that our system begins as
        \[
            \begin{pmatrix}
                \frac{7}{2} & 1 & 0 & \cdots & \cdots & 0 \\
                1 & 4 & 1 & 0 & \cdots & 0 \\
                0 & \ddots & \ddots & \ddots & \ddots & \\
                0 & \cdots & \cdots & 1 & 4 & 1 \\
                0 & \cdots & \cdots & \cdots & 1 & \frac{7}{2}
            \end{pmatrix}
            \begin{pmatrix}
                d_1 \\
                d_2 \\
                \vdots \\
                d_{N - 2} \\
                d_{N - 1}
            \end{pmatrix}
            =
            \begin{pmatrix}
                6x_1 - \frac{3}{2}d_0 \\
                6x_2 \\
                \vdots \\
                6x_{N - 2} \\
                6x_{N - 1} - \frac{3}{2}d_N \\
            \end{pmatrix}.
        \]
        We have been provided the points $d_0 = \frac{2}{3}x_0 + \frac{1}{3}d_1$
        and $d_N = \frac{1}{3}d_{N - 1} + \frac{2}{3}x_N$. Substituting these
        values in to our initial system, we get
        \begin{align*}
            \begin{pmatrix}
                \frac{7}{2} & 1 & 0 & \cdots & \cdots & 0 \\
                1 & 4 & 1 & 0 & \cdots & 0 \\
                0 & \ddots & \ddots & \ddots & \ddots & \\
                0 & \cdots & \cdots & 1 & 4 & 1 \\
                0 & \cdots & \cdots & \cdots & 1 & \frac{7}{2}
            \end{pmatrix}
            \begin{pmatrix}
                d_1 \\
                d_2 \\
                \vdots \\
                d_{N - 2} \\
                d_{N - 1}
            \end{pmatrix}
            & =
            \begin{pmatrix}
                6x_1 - \frac{3}{2}(\frac{2}{3}x_0 + \frac{1}{3}d_1) \\
                6x_2 \\
                \vdots \\
                6x_{N - 2} \\
                6x_{N - 1} - \frac{3}{2}(\frac{1}{3}d_{N - 1} + \frac{2}{3}x_N) \\
            \end{pmatrix}
            \\\\ & =
            \begin{pmatrix}
                6x_1 - x_0 - \frac{1}{2}d_1 \\
                6x_2 \\
                \vdots \\
                6x_{N - 2} \\
                6x_{N - 1} - \frac{1}{2}d_{N - 1} - x_N \\
            \end{pmatrix}.
        \end{align*}
        From here, we can multiply out the left side to get
        \begin{align*}
            \begin{pmatrix}
                \frac{7}{2}d_1 + d_2 \\
                d_1 + 4d_2 + 1d_3 \\
                \vdots \\
                d_{N - 3} + 4d_{N - 2} + d_{N - 1} \\
                d_{N - 2} + \frac{7}{2}d_{N - 1}
            \end{pmatrix}
            & =
            \begin{pmatrix}
                6x_1 - x_0 - \frac{1}{2}d_1 \\
                6x_2 \\
                \vdots \\
                6x_{N - 2} \\
                6x_{N - 1} - \frac{1}{2}d_{N - 1} - x_N \\
            \end{pmatrix}.
        \end{align*}
        We can then simplify our system
        \begin{align*}
            \begin{pmatrix}
                \frac{8}{2}d_1 + d_2 \\
                d_1 + 4d_2 + 1d_3 \\
                \vdots \\
                d_{N - 3} + 4d_{N - 2} + d_{N - 1} \\
                d_{N - 2} + \frac{8}{2}d_{N - 1}
            \end{pmatrix}
            & =
            \begin{pmatrix}
                6x_1 - x_0 \\
                6x_2 \\
                \vdots \\
                6x_{N - 2} \\
                6x_{N - 1} - x_N \\
            \end{pmatrix}.
            \\\\
            \begin{pmatrix}
                4d_1 + d_2 \\
                d_1 + 4d_2 + 1d_3 \\
                \vdots \\
                d_{N - 3} + 4d_{N - 2} + d_{N - 1} \\
                d_{N - 2} + 4d_{N - 1}
            \end{pmatrix}
            & =
            \begin{pmatrix}
                6x_1 - x_0 \\
                6x_2 \\
                \vdots \\
                6x_{N - 2} \\
                6x_{N - 1} - x_N \\
            \end{pmatrix}.
        \end{align*}
        Then, by expanding back out, we find
        \[
            \begin{pmatrix}
                4 & 1 & 0 & \cdots & \cdots & 0 \\
                1 & 4 & 1 & 0 & \cdots & 0 \\
                0 & \ddots & \ddots & \ddots & \ddots & \\
                0 & \cdots & \cdots & 1 & 4 & 1 \\
                0 & \cdots & \cdots & \cdots & 1 & 4
            \end{pmatrix}
            \begin{pmatrix}
                d_1 \\
                d_2 \\
                \vdots \\
                d_{N - 2} \\
                d_{N - 1}
            \end{pmatrix}
            =
            \begin{pmatrix}
                6x_1 - x_0 \\
                6x_2 \\
                \vdots \\
                6x_{N - 2} \\
                6x_{N - 1} - x_N \\
            \end{pmatrix}
        \]
        which is what we sought.

        To show that our system reduces to
        \[
            \begin{pmatrix}
                1 & 4 \\
                4 & 1 \\
            \end{pmatrix}
            \begin{pmatrix}
                d_1 \\
                d_2 \\
            \end{pmatrix}
            =
            \begin{pmatrix}
                6x_1 - x_0 \\
                6x_2 - x_3 \\
            \end{pmatrix}
        \]
        for $N = 3$, we can follow a very similar approach. Take our initial
        formula for $N = 3$
        \[
            \begin{pmatrix}
                \frac{7}{2} & 1 \\
                1 & \frac{7}{2} \\
            \end{pmatrix}
            \begin{pmatrix}
                d_1 \\
                d_2 \\
            \end{pmatrix}
            =
            \begin{pmatrix}
                6x_1 - \frac{3}{2}d_0 \\
                6x_2 - \frac{3}{2}d_3 \\
            \end{pmatrix}.
        \]
        We then sub in for $d_0 = \frac{2}{3}x_0 + \frac{1}{3}d_1$ and
        $d_N = d_3 = \frac{1}{3}d_2 + \frac{2}{3}x_3$ to get
        \begin{align*}
            \begin{pmatrix}
                \frac{7}{2} & 1 \\
                1 & \frac{7}{2} \\
            \end{pmatrix}
            \begin{pmatrix}
                d_1 \\
                d_2 \\
            \end{pmatrix}
            & =
            \begin{pmatrix}
                6x_1 - \frac{3}{2}(\frac{2}{3}x_0 + \frac{1}{3}d_1) \\
                6x_2 - \frac{3}{2}(\frac{1}{3}d_2 + \frac{2}{3}x_3) \\
            \end{pmatrix}
            \\ & =
            \begin{pmatrix}
                6x_1 - x_0 - \frac{1}{2}d_1 \\
                6x_2 - \frac{1}{2}d_2 - x_3 \\
            \end{pmatrix}
        \end{align*}
        Then by simplifying, we get
        \begin{align*}
            \begin{pmatrix}
                \frac{7}{2}d_1 + d_2 \\
                d_1 + \frac{7}{2}d_2 \\
            \end{pmatrix}
            & =
            \begin{pmatrix}
                6x_1 - x_0 - \frac{1}{2}d_1 \\
                6x_2 - \frac{1}{2}d_2 - x_3 \\
            \end{pmatrix}
            \\
            \begin{pmatrix}
                \frac{8}{2}d_1 + d_2 \\
                d_1 + \frac{8}{2}d_2 \\
            \end{pmatrix}
            & =
            \begin{pmatrix}
                6x_1 - x_0 \\
                6x_2 - x_3 \\
            \end{pmatrix}
            \\
            \begin{pmatrix}
                4d_1 + d_2 \\
                d_1 + 4d_2 \\
            \end{pmatrix}
            & =
            \begin{pmatrix}
                6x_1 - x_0 \\
                6x_2 - x_3 \\
            \end{pmatrix}.
        \end{align*}
        Finally, we can expand back out to get
        \begin{align*}
            \begin{pmatrix}
                4 & 1 \\
                1 & 4 \\
            \end{pmatrix}
            \begin{pmatrix}
                d_1 \\
                d_2 \\
            \end{pmatrix}
            & =
            \begin{pmatrix}
                6x_1 - x_0 \\
                6x_2 - x_3 \\
            \end{pmatrix}
        \end{align*}
        as desired.
    \end{proof}
\end{document}
